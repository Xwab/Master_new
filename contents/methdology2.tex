\chapter{层间动态Rank分配与误差调控}
\label{chap:dynamic-rank}

上一章我们已经给出了基于激活/注意力重加权的低秩分解算法,能够在每一层内部找到与真实推理目标 $XW_k$、$A X W_v$ 最匹配的主奇异子空间。本章的方法第二部分在此基础上进一步回答“每一层应该保留多少秩”这一跨层问题:我们不再为所有层设定统一的压缩率,而是同时建模(1)分解目标自身的低秩性,刻画该层在联合 SVD 下主能量的集中程度,并用作 rank 初始化的先验;(2)对丢弃秩的敏感性,评估 Key/Value 在注意力反馈中的梯度放大与残差重建难度,用于指导跨层 rank 传递时的步长;(3)多层串联时低秩近似误差的累积风险,确保深层段的近似误差不会在长链路 KV 缓存里失控,并在不同层之间传递 rank 配额时提供约束信号。通过将这三类指标归一化为层级打分,并在全模型 rank 预算约束下运行一个逐步迭代的贪心分配过程,我们可以为每一层分配自适应的秩上限,保证激活/注意力感知的分解结果有足够的表达度,同时把全局误差保持在推理精度可接受的范围。接下来将从指标构建、分配算法与训练/推理期间的更新策略三个角度展开层间 rank 分配的具体流程。

\section{基于参数低秩性的层间秩初始化}
低秩降维成立的基本前提是:所分解的目标矩阵本身在能量上呈现显著的低秩特性。我们的对象不是原始的 $W_k/W_v$,而是上一章通过变换矩阵 $S_{k,l}$、$S_{v,l}$ 吸收了激活与注意力统计后的 $M_{k,l}=S_{k,l}W_{k,l}$、$M_{v,l}=S_{v,l}W_{v,l}$。只有当这些矩阵在各自的 SVD 中表现出主奇异值快速衰减时,基于奇异值能量的 rank 初始化和裁剪才有意义。虽然 Key 与 Value 普遍呈现低秩特征,但能量集中程度会随层号与矩阵类别发生明显变化:不少层的 Key 在前几个秩内就累计到 80\% 以上,而同层的 Value 曲线往往更平缓,意味着需要保留更多秩才能达到同等能量。累积能量曲线正好把这种差异量化出来,为“哪些层、哪类矩阵可以优先裁剪尾部秩”提供先验。

跨层 rank 调度可以转写为一个归一化误差最小化问题。设共有 $L$ 个注意力层,第 $l$ 层在 GQA 架构下含有 $G_l$ 个 KV 组,Key/Value 特征维度统一记作 $d$。我们分别对 $M_{k,l}$ 与 $M_{v,l}$ 做 SVD,并保留 $r_{k,l}$、$r_{v,l}$ 个奇异值。若定义 $M_{k,l}^{(r_{k,l})}$、$M_{v,l}^{(r_{v,l})}$ 为对应的最优秩-$r$ 近似,则初始化阶段希望求解
\begin{equation}
    \label{eq:opt-init}
    \min_{\{r_{k,l},r_{v,l}\}} \sum_{l=1}^{L}
    \left(
        \frac{\bigl\|M_{k,l}-M_{k,l}^{(r_{k,l})}\bigr\|_F^2}{\|M_{k,l}\|_F^2}
      + \frac{\bigl\|M_{v,l}-M_{v,l}^{(r_{v,l})}\bigr\|_F^2}{\|M_{v,l}\|_F^2}
    \right),
\end{equation}
使得丢弃的总秩满足
\begin{equation}
    \label{eq:rank-budget}
    \sum_{l=1}^{L} \Bigl[(d-r_{k,l}) + (d-r_{v,l})\Bigr]
    = (1-\rho)\sum_{l=1}^{L} 2d,
\end{equation}
其中 $\rho$ 为目标保留比例。约束式意味着所有层被丢弃的 rank 数之和恰好等于 $(1-\rho)$ 倍的模型总 rank。

由于 SVD 的最优性,式 \eqref{eq:opt-init} 中的归一化误差可以直接用奇异值能量表示。记 $\{\sigma_{k,l,i}\}_{i=1}^{d}$、$\{\sigma_{v,l,i}\}_{i=1}^{d}$ 为降序排列的奇异值,则有
\begin{equation}
    \label{eq:energy-weight-kv}
    w_{k,l,i} = \frac{\sigma_{k,l,i}^2}{\sum_{j=1}^{d} \sigma_{k,l,j}^2},\qquad
    w_{v,l,i} = \frac{\sigma_{v,l,i}^2}{\sum_{j=1}^{d} \sigma_{v,l,j}^2},
\end{equation}
并满足 $\sum_i w_{k,l,i} = \sum_i w_{v,l,i} = 1$。则
\[
    \frac{\bigl\|M_{k,l}-M_{k,l}^{(r_{k,l})}\bigr\|_F^2}{\|M_{k,l}\|_F^2}
    = \sum_{i=r_{k,l}+1}^{d} w_{k,l,i}, \qquad
    \frac{\bigl\|M_{v,l}-M_{v,l}^{(r_{v,l})}\bigr\|_F^2}{\|M_{v,l}\|_F^2}
    = \sum_{i=r_{v,l}+1}^{d} w_{v,l,i}.
\]
因此,优化问题 \eqref{eq:opt-init} 等价于在约束 \eqref{eq:rank-budget} 下最大化各层被保留下来的奇异值能量之和:
\begin{equation}
    \max_{\{r_{k,l},r_{v,l}\}} \sum_{l=1}^{L}
    \left(
        \sum_{i=1}^{r_{k,l}} w_{k,l,i}
      + \sum_{i=1}^{r_{v,l}} w_{v,l,i}
    \right)
    \quad \text{s.t. } \eqref{eq:rank-budget}.
    \label{eq:max-energy}
\end{equation}
这表明:初始化时的最优策略就是优先保留所有 Key/Value 矩阵中能量占比最高的秩,把能量最小的奇异值作为候选裁剪对象,直到总共丢弃 $(1-\rho)$ 的秩。

实践中我们不需要显式求解 \eqref{eq:max-energy},而是把式 \eqref{eq:energy-weight-kv} 得到的能量权重按层内降序排序,并从尾部开始依次丢弃奇异值,直到满足约束 \eqref{eq:rank-budget} 为止。这一过程等价于在“所有奇异值能量”集合上执行一次全局的 top-$\rho$ 选择,既保证了全局 rank 预算,又忠实反映了不同层、不同类型矩阵的低秩程度,为后续考虑误差传递和敏感性调控的贪心迭代提供了有针对性的起点。下一节将在此基础上引入层间误差累积约束,对这些初值进一步细化。