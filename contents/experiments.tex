\chapter{实验设计与结果分析}
本章面向多个评测维度系统验证所提压缩方案的有效性:我们在语言建模(WikiText-2、PTB)、长上下文理解(LongBench 21 个子任务涵盖检索、推理与 Summarization)、阅读理解/问答(BoolQ、OpenBookQA、ARC-C、ARC-E)、常识与推理(HellaSwag、Winogrande、PIQA)以及标准化考试(CEval)等多种任务类型上运行实验,覆盖困惑度、Rouge、准确率等多项指标;同时选取 Llama3 系列与 Mixtral 等不同规模的大模型,在多档 KV 压缩率(从轻压缩到极限压缩)、多种输入长度(中短文本到十万级 token)下评估性能。为确保结论可靠,我们先使用前两章提出的“激活/注意力感知低秩分解 + 层间秩重分配”方案(未引入量化)与最强低秩压缩 baseline 做对比,再进一步叠加第三章提出的混合精度与可控粒度量化,展示在同一压缩率预算下性能能否继续提升。通过与当前表现最好的 KV 压缩与量化方法逐一对照,我们强调低秩感知的混合精度策略在复杂推理场景中的优势与可拓展性。