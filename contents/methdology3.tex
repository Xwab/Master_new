\chapter{低秩感知的混合精度多粒度量化策略}
完成层间秩分配只是降低 KV 缓存体积的第一步;若希望进一步压缩,就必须引入低比特量化。然而已有实验(见表~\ref{tab:value-quant-granularity})表明:当 Key/Value 被粗粒度地统一压到 2-bit 时,模型困惑度急剧上升,而 Value 分支因语义表达更敏感,退化尤为明显,提示我们需要比逐通道更细的量化粒度。其次,一旦把量化与低秩裁剪结合,层与层之间保留的 rank 彼此不同,group-wise 量化的组大小往往无法整除各自的特征维度,导致细粒度量化难以落地。最后,即使解决了粒度问题,我们仍需在不同秩之间区别对待量化精度:低秩分解所暴露的奇异值能量差异意味着重要秩必须维持较高精度,而长尾秩则可采用更低比特。如何在这三重约束下,把“粗/细粒度量化”与“混合精度、低秩感知”协同起来,构成了本章方法的出发点。

\begin{table}[htbp]
    \centering
    \caption{不同量化粒度在 2-bit 下的困惑度(Perplexity)对比}
    \label{tab:value-quant-granularity}
    \begin{tabular}{lcc}
        \toprule
        \textbf{量化粒度} & \textbf{WikiText-2} & \textbf{PTB} \\
        \midrule
        通道级(channel-wise, 2-bit) & 514.59 & 2665.66 \\
        分组级(group-wise, 2-bit) & 11.17 & 20.87 \\
        \bottomrule
    \end{tabular}
\end{table}