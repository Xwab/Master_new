\chapter{Value 感知的分组量化压缩策略}
在完成层间秩分配之后,我们进一步观察到 Value 分支的低秩性显著弱于 Key:若仍沿用单纯“丢弃尾部秩”的压缩方式,Value 侧往往会失去比 Key 更多的语义信息,同时主奇异值对应的秩本身也需要更高精度来保持能量。因此本章提出一种“保留全部秩、分层量化再分组压缩”的 Value 专属方案:对原本应保留的秩采用较高精度(如 8-bit)量化以稳住关键信息,而对原本计划丢弃的秩则不再删除,而是以更细粒度的低精度量化(如 4-bit)来保留长尾上下文贡献;与传统在低秩裁剪后再量化不同,我们通过先保留全部特征维度,再利用可增减量化组的方式保证任意层的特征维度都能整除 group-size,从而实现对不重要秩的更细粒度压缩,同时兼顾 Value 分支的信息完整性与存储开销。