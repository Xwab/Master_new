\chapter{Value 感知的分组量化压缩策略}
完成层间秩分配只是降低 KV 缓存体积的第一步;若希望进一步压缩,就必须引入低比特量化。然而已有实验(见表~\ref{tab:value-quant-granularity})表明:当 Key/Value 被粗粒度地统一压到 2-bit 时,模型困惑度急剧上升,而 Value 分支因语义表达更敏感,退化尤为明显,提示我们需要比逐通道更细的量化粒度。其次,一旦把量化与低秩裁剪结合,层与层之间保留的 rank 彼此不同,group-wise 量化的组大小往往无法整除各自的特征维度,导致细粒度量化难以落地。最后,即使解决了粒度问题,我们仍需在不同秩之间区别对待量化精度:重要秩应保留更多信息,轻量秩可以牺牲部分精度。如何在这三重约束下把量化与低秩压缩协同起来,构成了本章方法的出发点。